% Created 2017-12-02 Sat 17:05
\documentclass[11pt]{article}
\usepackage[utf8]{inputenc}
\usepackage[T1]{fontenc}
\usepackage{fixltx2e}
\usepackage{graphicx}
\usepackage{longtable}
\usepackage{float}
\usepackage{wrapfig}
\usepackage{rotating}
\usepackage[normalem]{ulem}
\usepackage{amsmath}
\usepackage{textcomp}
\usepackage{marvosym}
\usepackage{wasysym}
\usepackage{amssymb}
\usepackage{hyperref}
\tolerance=1000
\usepackage{xeCJK}
\setCJKmainfont{宋体}
\hypersetup{colorlinks=true,linkcolor=red}
\usepackage{minted}
\usepackage{geometry}
\geometry{left=2.0cm,right=2.0cm,top=2.5cm,bottom=2.5cm}
\setmainfont{Times New Roman}
\setsansfont{Arial}
\setmonofont{Courier New}
\author{MiracleEEEE}
\date{\today}
\title{Orgmode Note}
\hypersetup{
  pdfkeywords={},
  pdfsubject={},
  pdfcreator={Emacs 25.3.1 (Org mode 8.2.10)}}
\begin{document}

\maketitle
\tableofcontents

\section{笔记结构}
\label{sec-1}

\subsection{章节}
\label{sec-1-1}

在orgmode中,n个*表示n级章节,例如:
\begin{verbatim}
* 一级标题
** 二级标题
\end{verbatim}
快捷键:\\
S-Tab 展开,折叠所有章节\\
Tab 对当前章节进行折叠\\
M-left/right 升级,降级标题\\

\subsection{列表}
\label{sec-1-2}

列表可以列出所有待完成事项。orgmode可以加入checkbok "[ ]" 来标记任务的完成情况。\\
如果一个任务有多个子任务,还可以根据子任务的完成情况来计算总进度,只需要在总任务下面加上“[ \% ]”或者“[ / ]”(不包含空格)。\\
列表分为两种,有序列表以“1.”或者“1)”开头,无序列表以“+”或者“-”开头。同样,后面要跟一个空格。\\
\begin{verbatim}
+ root
  + branch1
  + branch2

1) [ ] root1 [%]
  1) [ ] branch1
  2) [ ] branch2
2) [ ] root2
\end{verbatim}
快捷键:\\
M-RET 插入统计列表项\\
M-S-RET 插入一个带checkbox的列表项\\
C-c C-c 改变checkbox的状态\\
M-left/right 改变列表项的层级关系\\
M-up/down 上下移动列表项\\

\subsection{脚注}
\label{sec-1-3}

\begin{verbatim}
插入脚注采用[fn:1]的方式,在最下面插入[fn:1]OrgMode-Note。这个标签是可以点击的。
\end{verbatim}

\subsection{表格}
\label{sec-1-4}

orgmode提供的方便的表格操作。最独特的是支持类似Excel的表格函数,可以完成求和等操作。\\
创建表格时,首先输入表头:
\begin{verbatim}
| Name | Phone | sub1 | sub2 | total|
|-
\end{verbatim}
然后按下tab,表格就会自动生成。然后可以输入数据,再输入的时候,按tab可以跳到右方表格,按enter能跳到下方表格。按住shift则反向跳。输入完成后,按C-c C-c可以对齐表格。
\begin{verbatim}
| Name |  Phone | sub1 | sub2 | total |
|------+--------+------+------+-------|
| Tom  | 134... |   89 |   98 |       |
| Jack | 152... |   78 |   65 |       |
| Ken  | 123... |   76 |   87 |       |
| Ana  | 157... |   87 |   78 |       |
\end{verbatim}
对于表格函数,我们在total列选择一个位置,然后输入
\begin{verbatim}
=$3+$4
\end{verbatim}
按下C-u C-c C-c,orgmode就能计算所有的第三列加第四列的和,并放到第五列。
\begin{center}
\begin{tabular}{llrrr}
Name & Phone & sub1 & sub2 & total\\
\hline
Tom & 134\ldots{} & 89 & 98 & 187\\
Jack & 152\ldots{} & 78 & 65 & 143\\
Ken & 123\ldots{} & 76 & 87 & 163\\
Ana & 157\ldots{} & 87 & 78 & 165\\
\end{tabular}
\end{center}
如果要插入行和列,可以在表头添加一个标签或者新起一行,再输入|调整格式即可。其中最后一行是orgmode自动添加的。\\
快捷键:\\
C-c | 通过输入大小的方式插入表格\\
C-c C-c 对齐表格\\
tab 跳到右边一个表格 \\
enter 跳到下方的表格 \\
M-up/right/left/right 上下左右移动行(列)\\
M-S-up/right/left/right 向上下左右插入行(列)\\

\subsection{链接}
\label{sec-1-5}

用于链接一些资源地址,比如文件,图片,URL等。链接的格式是:
\begin{verbatim}
[[链接地址][链接内容]]

如:
[[http://orgmode.org/orgguide.pdf][grgguid.pdf]]]
[[file:/home/maple/图片/test.jpg][a picture]]

如果去掉标签,则能直接显示图片:
[[file:/home/maple/图片/test.jpg]]
\end{verbatim}
直接显示的图片在Emacs默认不显示,需要按C-c C-x C-v才能显示。\\
快捷键:\\
C-c C-x C-v 预览图片\\

\subsection{待办事项(TODO)}
\label{sec-1-6}

TODO是orgmode最有特色的一个功能,可以完成一个GDT。\\
TODO也是一类标题,因此也需要“*”开头。
\begin{verbatim}
** TODO 刷题
\end{verbatim}
初始TODO为红色,如果将光标移动到该行并按下C-c C-t,则发现TODO变成了DONE。
\begin{verbatim}
*** TODO [# A] 任务1
*** TODO [# B] 任务2
*** TODO 总任务 [33%]
**** TODO 子任务1
**** TODO 子任务2 [0%]
     - [-] subsub1 [1/2]
      - [ ] subsub2
      - [X] subsub3
**** DONE 一个已完成的任务
\end{verbatim}
快捷键:\\
C-c C-t 变换TODO的状态 \\
C-c / t 以树的形式展示所有的 TODO \\
C-c C-c 改变 checkbox状态 \\
C-c , 设置优先级(方括号里的ABC) \\
M-S-RET 插入同级TODO标签\\

\section{标记}
\label{sec-2}

\subsection{标签Tags}
\label{sec-2-1}

在orgmode中,可以给每一章添加一个标签。可以通过树的结构来查看带标签的章节。在每一节中,子标题的标签会继承父标题的标签。
\begin{verbatim}
*** 章标题                                                       :work:learn:
**** 节标题1                                                      :fly:plane:
**** 节标题2                                                        :car:run:
\end{verbatim}
快捷键:\\
C-c C-q 为标题添加标签 \\
C-c / m 生成带标签的树 \\

\subsection{时间}
\label{sec-2-2}

orgmode可以利用emacs的时间插入当前的时间。输入C-c . 会出现一个日历,选择相应的时间插入即可。\textit{<2017-12-01 Fri>} \\
时间可以添加DEADLINE和SCHEDULED表示时间的类型。
\begin{verbatim}
DEADLINE:<2017-12-01 Fri>
\end{verbatim}
快捷键: \\
C-c . 插入时间 \\

\subsection{特殊文本格式}
\label{sec-2-3}

\begin{verbatim}
*bold*
/italic/
_underlined_
=code=
~verbatim~
+strike-through+
\end{verbatim}

\subsection{富文本导出}
\label{sec-2-4}

主要用于导出pdf或者html时制定导出选项。 \\

\subsubsection{设置标题和目录}
\label{sec-2-4-1}

\begin{verbatim}
# +TITLE: This is the title of the document
# +OPTIONS: toc:2 (只显示两级目录)
# +OPTIONS: toc:nil (不显示目录)
\end{verbatim}

\subsubsection{添加引用}
\label{sec-2-4-2}

\begin{verbatim}
#+BEGIN_QUOTE
Everything should be made as simple as possible,
but not any simpler -- Albert Einstein
#+END_QUOTE
\end{verbatim}

\begin{quote}
Everything should be made as simple as possible,
but not any simpler -- Albert Einstein
\end{quote}
快捷键:\\
键入 <q 之后按下 tab 自动补全。

\subsubsection{设置居中}
\label{sec-2-4-3}

\begin{verbatim}
#+BEGIN_CENTER
Everything should be made as simple as possible,but not any simpler
#+END_CENTER
\end{verbatim}

\begin{center}
Everything should be made as simple as possible,but not any simpler
\end{center}
快捷键:\\
键入 <c 之后按下 tab 自动补全。

\subsubsection{设置样例}
\label{sec-2-4-4}

\begin{verbatim}
实际应该为BEGIN_EXAMPLE和END_EXAMPLE

#+BEGINEXAMPLE
这里面的字符不会被转义。
#+ENDEXAMPLE
\end{verbatim}
快捷键:\\
键入 <e 之后按下 tab 自动补全。
\subsubsection{注释}
\label{sec-2-4-5}

\begin{verbatim}
# comment
或者:
#+BEGIN_COMMENT
这里的注释不会被导出
#+END_COMMENT
\end{verbatim}

\subsubsection{LATEX}
\label{sec-2-4-6}

\begin{verbatim}
嵌入公式:\( \) 或 $ $
行间公式:\[ \] 或 $$ $$
\end{verbatim}
LATEX能支持直接输入LATEX。

$$ ax+by+c $$
快捷键:\\
C-c C-x C-l 预览LATEX图片。\\

\subsubsection{源代码}
\label{sec-2-4-7}

\begin{verbatim}
#+BEGIN_SRC C++
#include <cstdio>
using namespace std;
int main() {
  int a=1;
  int b=1;
  printf("%d\n",a+b);
}

#+END_SRC
\end{verbatim}

\begin{minted}[]{c++}
#include <cstdio>
using namespace std;
int main() {
  int a=1;
  int b=1;
  printf("%d\n",a+b);
}
\end{minted}
快捷键:\\
C-c C-c 对当前代码块求值

\subsubsection{文章信息}
\label{sec-2-4-8}
\begin{verbatim}
#+TITLE:       the title to be shown (default is the buffer name)
#+AUTHOR:      the author (default taken from user-full-name)
#+DATE:        a date, fixed, of a format string for format-time-string
#+EMAIL:       his/her email address (default from user-mail-address)
#+DESCRIPTION: the page description, e.g. for the XHTML meta tag
#+KEYWORDS:    the page keywords, e.g. for the XHTML meta tag
#+LANGUAGE:    language for HTML, e.g. ‘en’ (org-export-default-language)
#+TEXT:        Some descriptive text to be inserted at the beginning.
#+TEXT:        Several lines may be given.
#+OPTIONS:     H:2 num:t toc:t \n:nil @:t ::t |:t ^:t f:t TeX:t ...
#+LINK_UP:     the ``up'' link of an exported page
#+LINK_HOME:   the ``home'' link of an exported page
#+LATEX_HEADER: extra line(s) for the LaTeX header, like \usepackage{xyz}
\end{verbatim}
\section{导出}
\label{sec-3}

C-c C-e 选择导出样式。

\subsection{PDF}
\label{sec-3-1}

PDF格式的导出需要先导出为LATEX文件然后再编译为PDF文件。\\
想要设置导出的页面大小的话,需要修改
\begin{verbatim}
\documentclass[a4paper]{article}
\end{verbatim}
如果PDF文件中含有中文,需要更改编译器为Xelatex然后在头文件中的$\backslash$documentclass下方加入\\
\begin{verbatim}
\usepackage{xeCJK}
\setCJKmainfont{宋体}
\end{verbatim}
如果需要插入代码,需要在头文件中加入\\
\begin{verbatim}
\usepackage{minted}
\end{verbatim}
页边距的设置:
\begin{verbatim}
\usepackage{geometry}
\geometry{left=2.0cm,right=2.0cm,top=2.5cm,bottom=2.5cm}
\end{verbatim}
去掉目录红边:
\begin{verbatim}
\hypersetup{colorlinks=true,linkcolor=red}
\end{verbatim}
字体设置:
\begin{verbatim}
\setmainfont{Times New Roman}
\setsansfont{Arial}
\setmonofont{Courier New}
\end{verbatim}
\section{参考资料}
\label{sec-4}

\href{https://github.com/marboo/orgmode-cn/blob/master/org.org}{Org-Manual 7.8}
% Emacs 25.3.1 (Org mode 8.2.10)
\end{document}
