% Created 2017-12-11 一 19:55
% Intended LaTeX compiler: pdflatex
\documentclass[11pt]{article}
\usepackage[utf8]{inputenc}
\usepackage[T1]{fontenc}
\usepackage{fixltx2e}
\usepackage{graphicx}
\usepackage{longtable}
\usepackage{float}
\usepackage{wrapfig}
\usepackage{rotating}
\usepackage[normalem]{ulem}
\usepackage{amsmath}
\usepackage{textcomp}
\usepackage{marvosym}
\usepackage{wasysym}
\usepackage{amssymb}
\usepackage{hyperref}
\tolerance=1000
\usepackage{xeCJK}
\setCJKmainfont{Noto Serif CJK KR SemiBold}
\hypersetup{colorlinks=true,linkcolor=red}
\usepackage{minted}
\usepackage{geometry}
\geometry{left=2.0cm,right=2.0cm,top=2.5cm,bottom=2.5cm}
\usepackage{indentfirst}
\setlength{\parindent}{2em}
\author{MiracleEEEE}
\date{\today}
\title{Task}
\hypersetup{
 pdfauthor={MiracleEEEE},
 pdftitle={Task},
 pdfkeywords={},
 pdfsubject={},
 pdfcreator={Emacs 27.0.50 (Org mode 9.1.4)}, 
 pdflang={English}}
\begin{document}

\maketitle
\tableofcontents


\section{知识点补全计划}
\label{sec:org547ca53}
\subsection{省选 [0/7]}
\label{sec:orge862b86}
\subsubsection{{\bfseries\sffamily TODO} 字符串 [4/6]}
\label{sec:orgf6997c7}
\begin{itemize}
\item $\boxtimes$ 后缀数组
\item $\square$ 后缀自动机
\item $\square$ 后缀平衡树
\item $\boxtimes$ AC自动机
\item $\boxtimes$ KMP及扩展KMP
\item $\boxtimes$ manacher
\end{itemize}
\subsubsection{{\bfseries\sffamily TODO} 图论 [3/17]}
\label{sec:org037b9a8}

\begin{itemize}
\item $\boxtimes$ 双连通分量
\item $\square$ 最大流
\item $\square$ 费用流
\item $\square$ 最小割
\item $\square$ 带上下界的网络流
\item $\boxtimes$ 树剖
\item $\square$ LCT
\item $\boxtimes$ 点分治
\item $\square$ 边分治
\item $\square$ 动态树分治
\item $\square$ 树分块
\item $\square$ 虚树
\item $\square$ 仙人掌
\item $\square$ 朱刘算法
\item $\square$ 弦图
\item $\square$ 区间图
\item $\square$ 对偶图
\end{itemize}

\subsubsection{{\bfseries\sffamily TODO} 数学 [4/25]}
\label{sec:org09b98e8}

\begin{itemize}
\item $\boxtimes$ 中国剩余定理
\item $\square$ 博弈论
\item $\square$ 拉格朗日乘子法
\item $\square$ 单纯型
\item $\square$ 辛普森积分
\item $\boxtimes$ 容斥原理
\item $\square$ 莫比乌斯反演
\item $\square$ BSGS
\item $\square$ 置换群
\item $\square$ FFT
\item $\square$ NTT
\item $\square$ 多项式求逆
\item $\square$ 二次剩余
\item $\square$ 多项式科技
\item $\square$ 积分
\item $\square$ 极限
\item $\square$ 微分
\item $\square$ 导数
\item $\square$ Ploya定理
\item $\square$ 贝叶斯定理
\item $\square$ 杜教筛
\item $\square$ Pollard-Rho 圆锥曲线分解法
\item $\boxtimes$ 线性基
\item $\square$ Miller-Rabin 素性探测
\item $\boxtimes$ 高斯消元
\end{itemize}

\subsubsection{{\bfseries\sffamily TODO} 动态规划 [2/6]}
\label{sec:orga7cf2df}

\begin{itemize}
\item $\square$ 斜率优化
\item $\square$ 插头DP
\item $\square$ 四边形不等式
\item $\square$ 斯坦纳树
\item $\boxtimes$ 数位DP
\item $\boxtimes$ 区间DP
\end{itemize}

\subsubsection{{\bfseries\sffamily TODO} 计算几何 [2/9]}
\label{sec:org3fb9a41}

\begin{itemize}
\item $\boxtimes$ 基础内容
\item $\square$ 凸包
\item $\square$ 三角剖分
\item $\square$ 旋转卡壳
\item $\square$ 半平面交
\item $\square$ picks定理
\item $\boxtimes$ 扫描线
\item $\square$ 动态凸包
\item $\square$ 三维计算几何
\end{itemize}

\subsubsection{{\bfseries\sffamily TODO} 搜索 [0/3]}
\label{sec:orgb8991fe}

\begin{itemize}
\item $\square$ 模拟退火
\item $\square$ 爬山算法
\item $\square$ 随机增量法
\end{itemize}

\subsubsection{{\bfseries\sffamily TODO} 数据结构 [0/4]}
\label{sec:org7335670}

\begin{enumerate}
\item {\bfseries\sffamily TODO} 离线算法 [1/5]
\label{sec:org110af41}

\begin{itemize}
\item $\square$ 莫队
\item $\square$ 树上莫队
\item $\square$ 单调莫队
\item $\square$ CDQ分治
\item $\boxtimes$ 整体二分
\end{itemize}

\item {\bfseries\sffamily TODO} 平衡树 [1/3]
\label{sec:org26bed65}

\begin{itemize}
\item $\square$ rope
\item $\boxtimes$ Treap
\item $\square$ 替罪羊树
\end{itemize}

\item {\bfseries\sffamily TODO} 其他 [1/6]
\label{sec:orge6578ac}

\begin{itemize}
\item $\boxtimes$ 主席树
\item $\square$ 线段树
\item $\square$ 划分树
\item $\square$ KD-Tree
\item $\square$ 块状链表
\item $\square$ 二维线段树
\end{itemize}

\item {\bfseries\sffamily TODO} 可持久化数据结构 [0/5]
\label{sec:org5cdcd10}

\begin{itemize}
\item $\square$ 平衡树
\item $\square$ 数组
\item $\square$ Trie树
\item $\square$ 块状链表
\item $\square$ 动态仙人掌
\end{itemize}
\end{enumerate}

\section{鏼题计划}
\label{sec:org4780a72}
\subsection{题目泛做}
\label{sec:orgef21313}
\subsubsection{NOIP 题目泛做 [1/1]}
\label{sec:orgacff8be}
\begin{enumerate}
\item {\bfseries\sffamily DONE} \href{https://www.luogu.org/problemnew/show/2831}{NOIP 2016 愤怒的小鸟} \textit{<2017-11-24 五>}\hfill{}\textsc{状态压缩:动态规划}
\label{sec:org04c99ae}

\textbf{Description}

给出\(n \leq 18\)个敌人坐标,每次可以可以消灭一条过\((0,0)\)抛物线上的敌人,求最小次数。

\textbf{Solution}

\(n\)很小考虑状压。最朴素的动态规划,用\(f[s]\)表示消灭\(s\)中的敌人的方案数。枚举下一次消灭哪两个敌人,计算抛物线转移,时间复杂度\(O(n^32^n)\),还可以继续优化。我们可以预处理抛物线能消灭哪些敌人,时间复杂度变为\(O(n^22^n)\)。但还不够。考虑第一个敌人在这个状态转移的状态中一定被某一条抛物线消灭,这样我们只考虑过第一个敌人的抛物线,枚举其他敌人转移,一定不会丢失最优解。时间复杂度变为\(O(n2^n)\)。
\end{enumerate}

\subsection{杂}
\label{sec:orgc513feb}
\subsubsection{2017年11月 [2/2]}
\label{sec:orga87abf4}
\begin{enumerate}
\item {\bfseries\sffamily DONE} \href{https://vjudge.net/problem/POJ-3693}{POJ 3693 Maximum repetition substring} \textit{<2017-11-24 五>}\hfill{}\textsc{后缀数组:ST表}
\label{sec:org84d4142}

\textbf{Description}

给出一个字符串,求最大重复子串(重复次数最多,如果存在多个,求字典序最小的那一个)。

\textbf{Solution}

后缀数组的应用。直接下手不好解决,不妨枚举一下循环节的长度\(|L|\)。我们发现,任何一个循环节为\(|L|\)重复子串总会包含至少两个\(s[0],s[|L|],s[2|L|], \cdots\)字符。那么考虑枚举两个相邻的上述字符,可以通过后缀数组\(+ST\)表\(O(1)\)求出\(LCP\)的长度,但是最长公共子串的开头并不一定是我们枚举的字符,所以还需要求出最长向前能匹配多少。这可以通过倒过来做一次后缀数组得到。那么我们现在有了一个极长区间,可以求得这个区间的循环节个数\(k\),也就可以求出一个区间\([l,r]\)满足开头落在这个区间内部的最大重复子串的循环节个数都为\(k\)。只需要找字典序最小的一个。那么用\(ST\)表查一下这个区间内最小的\(rank\)的后缀就好了。时间复杂度\(O\Big(\sum_{i=1}^{n}\frac{n}{i}\Big)=O(nlogn)\)。有一个优化,当求得一个极长重复子串之后,落在子串内的\(s[i|L|]\)都不用再枚举了。

\item {\bfseries\sffamily DONE} \href{http://www.lydsy.com/JudgeOnline/problem.php?id=4310}{BZOJ 4310 跳蚤} \textit{<2017-11-29 三>}\hfill{}\textsc{后缀数组:ST表:二分查找}
\label{sec:org1d68e8d}

\textbf{Description}

给出一个字符串 \(S\) ,将它分成不超过 \(k\) 个子串,对于每个子串 \(T\) ,设 \(T'\) 为其最大子串,最小化选出的 \(k\) 个 \(T'_i\) 的最大值。

\textbf{Solution}

最大值最小可以二分。可以利用后缀数组求出本质不同的子串个数,对于一个第 \(k\) 大的子串,可以利用后缀数组求出它具体是谁。然后从后向前贪心扫描分块即可。注意一下比较两个串大小时的细节。设第 \(k\) 大的子串和要比较的串在后缀数组中第一次出现的位置分别为为 \(p_0,p_1\) ,分 \(p_1 < p_0,p_1=p_0,p_1 > p_0\) 三种情况讨论即可。
\end{enumerate}
\subsubsection{2017年12月 [2/2]}
\label{sec:orgbf40e97}
\begin{enumerate}
\item {\bfseries\sffamily DONE} \href{http://www.lydsy.com/JudgeOnline/problem.php?id=3514}{BZOJ 3514 Codechef MARCH14 GERALD07加强版} \textit{<2017-12-06 三>}\hfill{}\textsc{LCT:主席树:贪心}
\label{sec:orga110002}

\textbf{Description}

给出一张无向图,每次询问边标号在\([l,r]\)区间内的子图的联通块个数。\(N \leq 200000\),强制在线。

\textbf{Solution}

考虑从特殊入手,如果是一棵给出的是一颗树,显然每次询问的答案是\(n-(r-l+1)\),图相比较树的区别是可能会出现环。构成环的边对答案是没有贡献的。如果我们能将每次询问对答案没有贡献的边都找出来,那么就解决了问题。按照编号从小到大依次加边,如果出现了一个环,不妨贪心的将编号最小的那一条边去掉(去掉最小的边不会影响答案!),设最小的那条边的编号为\(k\),如果\(k\)在\([l,r]\)之间,那么现在加的这条边是没有用的,否则如果\(k < l\),那么这条边有用,对答案有\(-1\)的贡献。也就是说我们需要求出加入每一条边之后去掉的边是哪一条,可以用一颗\(LCT\)来维护。对于查询,也就是查区间内小于\(l\)的数字个数,直接上主席树就好了。

\item {\bfseries\sffamily DONE} \href{https://arc086.contest.atcoder.jp/tasks/arc086\_b}{Atcoder ARC086 D Non-decreasing} \textit{<2017-12-11 一>}\hfill{}\textsc{构造}
\label{sec:orga32d33e}

\textbf{Description}

给出一个序列\(a_n\),每次可以执行操作\(a_j += a_i\),构造方案使得操作次数\(\leq 2n\)且数列单调不降。

\textbf{Solution}

不妨从特殊开始考虑,如果这个序列都为正数,那么显然只需要做一次前缀和就满足要求了。现在问题是数列不一定都是正数,那么只需要对每个数加上最大值即可(负数的情况类似讨论)。

\item {\bfseries\sffamily DONE} \href{https://arc086.contest.atcoder.jp/tasks/arc086\_c}{Atcoder ARC086 E Smuggling Marbles} \textit{<2017-12-11 一>}\hfill{}\textsc{树形DP:队列:计数}
\label{sec:org1253cb0}

\textbf{Description}

有一棵树,根为\(0\),每个点可能会有一颗石子,重复进行如下操作:

\begin{enumerate}
\item 如果\(0\)号节点有一颗石子,那么把这颗石子放到盒子里
\item 把每个石子移项它的父亲节点
\item 如果有一个节点的石子数量超过\(1\)个,那么删除这些石子。
\end{enumerate}

问对于所有的\(2^n\)种起始情况,最后盒子里面的石子总数是多少。

\textbf{Solution}

每一层都是独立的,考虑枚举深度,用\(f[i][j] (0 \leq j \leq 1)\)表示节点\(i\)有\(j\)个石子的方案数,进行树形DP枚举它的哪个孩子给它\(1\)的贡献转移。但是这样是\(O(n^2)\)的。

但是其实每个动态规划的线程都是独立的,不妨一起进行,用双端队列维护状态\(f[i][j][k] (0 \leq k \leq 2)\),\(k\)的三维可以用一个三元组维护,表示相对结点\(i\)的深度为\(j\)的结点对\(i\)贡献\(k\)个石子的方案数(\(k=2\)时表示有大于等于\(2\)个石子,这个状态是有必要维护的,这是与算法一的区别,因为转移的时候我们没有办法利用补集转移)。合并的时候就把小的队列与大的合并。只有深度相同的点合并才会对复杂度在\(LCA\)处产生贡献,而这些深度相同的点,设有\(x_i\)个,\(LCA\)总数是\(x_i-1\)。那么从的复杂度为\(O(\sum x_i)=O(n)\)。
\end{enumerate}
\end{document}
